\documentclass[12pt,dvipdfmx]{beamer}
\usepackage{graphicx}
\DeclareGraphicsExtensions{.pdf}
\DeclareGraphicsExtensions{.eps}
\graphicspath{{out/}{out/tex/}{out/tex/gpl/}{out/tex/svg/}{out/tex/dot/}}
% \graphicspath{{out/}{out/tex/}{out/pdf/}{out/eps/}{out/tex/gpl/}{out/tex/svg/}{out/pdf/dot/}{out/pdf/gpl/}{out/pdf/img/}{out/pdf/odg/}{out/pdf/svg/}{out/eps/dot/}{out/eps/gpl/}{out/eps/img/}{out/eps/odg/}{out/eps/svg/}}
\usepackage{listings}
\usepackage{fancybox}
\usepackage{hyperref}
\usepackage{color}

%%%%%%%%%%%%%%%%%%%%%%%%%%%
%%% themes
%%%%%%%%%%%%%%%%%%%%%%%%%%%
\usetheme{Szeged} 
%% no navigation bar
% default boxes Bergen Boadilla Madrid Pittsburgh Rochester
%% tree-like navigation bar
% Antibes JuanLesPins Montpellier
%% toc sidebar
% Berkeley PaloAlto Goettingen Marburg Hannover Berlin Ilmenau Dresden Darmstadt Frankfurt Singapore Szeged
%% Section and Subsection Tables
% Copenhagen Luebeck Malmoe Warsaw

%%%%%%%%%%%%%%%%%%%%%%%%%%%
%%% innerthemes
%%%%%%%%%%%%%%%%%%%%%%%%%%%
% \useinnertheme{circles}	% default circles rectangles rounded inmargin

%%%%%%%%%%%%%%%%%%%%%%%%%%%
%%% outerthemes
%%%%%%%%%%%%%%%%%%%%%%%%%%%
% outertheme
% \useoutertheme{default}	% default infolines miniframes smoothbars sidebar sprit shadow tree smoothtree


%%%%%%%%%%%%%%%%%%%%%%%%%%%
%%% colorthemes
%%%%%%%%%%%%%%%%%%%%%%%%%%%
\usecolortheme{seahorse}
%% special purpose
% default structure sidebartab 
%% complete 
% albatross beetle crane dove fly seagull 
%% inner
% lily orchid rose
%% outer
% whale seahorse dolphin

%%%%%%%%%%%%%%%%%%%%%%%%%%%
%%% fontthemes
%%%%%%%%%%%%%%%%%%%%%%%%%%%
\usefonttheme{serif}  
% default professionalfonts serif structurebold structureitalicserif structuresmallcapsserif

%%%%%%%%%%%%%%%%%%%%%%%%%%%
%%% generally useful beamer settings
%%%%%%%%%%%%%%%%%%%%%%%%%%%
% 
\AtBeginDvi{\special{pdf:tounicode EUC-UCS2}}
% do not show navigation
\setbeamertemplate{navigation symbols}{}
% show page numbers
\setbeamertemplate{footline}[frame number]

%%%%%%%%%%%%%%%%%%%%%%%%%%%
%%% define some colors for convenience
%%%%%%%%%%%%%%%%%%%%%%%%%%%

\newcommand{\mido}[1]{{\color{green}#1}}
\newcommand{\mura}[1]{{\color{purple}#1}}
\newcommand{\ore}[1]{{\color{orange}#1}}
\newcommand{\ao}[1]{{\color{blue}#1}}
\newcommand{\aka}[1]{{\color{red}#1}}

\setbeamercolor{ex}{bg=cyan!20!white}

%%%%%%%%%%%%%%%%%%%%%%%%%%%
%%% how to typset code
%%%%%%%%%%%%%%%%%%%%%%%%%%%

\lstset{language = C,
numbers = left,
numberstyle = {\tiny \emph},
numbersep = 10pt,
breaklines = true,
breakindent = 40pt,
frame = tlRB,
frameround = ffft,
framesep = 3pt,
rulesep = 1pt,
rulecolor = {\color{blue}},
rulesepcolor = {\color{blue}},
flexiblecolumns = true,
keepspaces = true,
basicstyle = \ttfamily\scriptsize,
identifierstyle = ,
commentstyle = ,
stringstyle = ,
showstringspaces = false,
tabsize = 4,
escapechar=\@,
}

\title{大規模ソフトウェアを手探る}
\institute{}
\author{田浦 \\ 細川颯介 (田浦研M1), 高橋淳一郎 (田浦研B4) }
\date{}

\AtBeginSubsection[] % Do nothing for \section*
{
\begin{frame}
\frametitle{Contents}
\tableofcontents[currentsection,currentsubsection]
\end{frame}
}

\begin{document}
\maketitle

%%%%%%%%%%%%%%%%%%%%%%%%%%%%%%%%%% 
\iffalse
\begin{frame}
\frametitle{Contents}
\tableofcontents
\end{frame}
\fi

%%%%%%%%%%%%%%%%% 
\begin{frame}
\frametitle{この課題の目標}
\begin{columns}
\begin{column}{0.4\textwidth}
  \begin{itemize}
  \item 「演習レベルの小さなプログラムが作れること」と,
    「実用規模のプログラムが作れること」の
    ギャップを埋める(ための知識と経験を得る)
  \item ささやかでも,ソフトをどう改良するかについて,
    アイデアを出しあう(whatとhow)
  \end{itemize}
\end{column}
\begin{column}{0.6\textwidth}
\begin{center}
\includegraphics[height=0.8\textheight]{out/pdf/svg/goal.pdf}
\end{center}
\end{column}
\end{columns}
\end{frame}

%%%%%%%%%%%%%%%%% 
\begin{frame}
\frametitle{演習レベル vs. 実用ソフト}
\begin{itemize}
\item 全く違うものか? $\rightarrow$ NO 
  (別に,特別な工場が必要な訳ではない\ldots)

\item しかし,
  \mura{演習をこなしていくだけでは,足りないものは確かにある}
\end{itemize}
\end{frame}

%%%%%%%%%%%%%%%%% 
\begin{frame}
\frametitle{違い}
\begin{itemize}
\item 表面的
{\small
\begin{tabular}{|p{2.5cm}|p{2.5cm}|p{4.5cm}|}\hline
項目 & 演習レベル & 実用ソフト \\\hline
ソースファイル & 1-数ファイル & 
多数ファイル/ディレクトリ \\
ビルド方法 & gcc一回 & makeその他自動化ツール \\
依存項目 & OS機能$+$標準ライブラリ関数
& OS機能$+$多数の外部ライブラリ \\
移植性 & 不要 & 要 (ifdef, configure, ...) \\
拡張性 & 不要 & 要 \\
汎用性 & そこまで不要 & 要 (コマンドライン, 設定ファイル, 環境変数, \ldots) \\\hline
\end{tabular}}

\item 質的:
ある程度以上の規模のソフトは,
\mura{「全てを隅々まで把握して」
  一行一行を書くわけではない}
\end{itemize}
\end{frame}

%%%%%%%%%%%%%%%%% 
\begin{frame}
\frametitle{鍵となる「スキル」}
\begin{itemize}
\item<1-> 全容がよくわからないソフトウェアの概要を把握し,
  拡張・変更できるようになる
\item<2-> $\leftarrow$ そのために,
  \ao{ソースファイル中で修正が必要な場所}を突き止める
\item<3-> $\leftarrow$ そのためのツール
  \begin{itemize}
  \item 初級: ソース検索(grepとか) ($\leftarrow$ これ「だけ」ってのは卒業したい\ldots)
  \item 中級: \ao{クロスリファレンスツール(gloalsとか)}
  \item 中級: \ao{デバッガ (gdb, lldb)}
  \item 中級: \ao{プロファイラ, トレーサ (uftrace)}
  \end{itemize}

\end{itemize}
\end{frame}

%%%%%%%%%%%%%%%%% 
\begin{frame}
\frametitle{付随する「基礎知識」}
\begin{itemize}
\item<1-> 多くのソフトをソースからビルドする,
  決まりきったやり方
  (make, configure, その他の,\mura{ビルドの自動化}ツール)
\item<2-> 多数のファイルからなるソフトはどう書くのか
  (C言語,\mura{分割コンパイル}の基本)
\item<3-> 実用ソフトなら常識的なソースの書きかたあれこれ
  \begin{itemize}
  \item<4-> \mura{コマンドライン引数,設定ファイル}など,
    プログラムを汎用的にするための書き方
  \item<5-> \mura{\tt \#ifdef}など,
    単一ソースを何通りにもコンパイルできるようにするための
    C言語の頻出処理
  \item<6-> その他,後々の修正がしやすいように,
    大きなプログラムがよくやっている技法
  \end{itemize}
\end{itemize}
\end{frame}

%%%%%%%%%%%%%%%%% 
\begin{frame}
\frametitle{これを身につけることの意義}
\begin{itemize}
\item ソフトウェアの研究では,
  \mura{成果をソフトウェアとして発信}することが重要

\item しかし大きなソフトウェアを一から作ることは,
  困難な場合が多い

\item $\rightarrow$ 既存のソフトウェアの拡張として
  作ることが多い

\item 一から作る場合でも,
  成果を実用的なソフトウェアとして発信する際,
  常識的なお作法を守っておくことは重要
\end{itemize}
\end{frame}

%%%%%%%%%%%%%%%%% 
\begin{frame}
\frametitle{課題全体のロードマップ}
\begin{enumerate}
\item デバッガでのソフトウェアの動作追跡手法を練習(本日)
\item チームを作る // 手探るソフトを決める
\item 役にたたなくてもいいから変更する案を決める(ミニ発表,議論)
\item 案に従って変更してみる(適宜,進捗発表)
\item なるべくやる気の起きる($\approx$役に立つ)
  変更目標を議論して決める
\item 目標に向かって実行
\item 最終発表
\end{enumerate}
\end{frame}

%%%%%%%%%%%%%%%%% 
\begin{frame}
\frametitle{オススメの心構え}
\begin{itemize}
\item<1-> この課題では,きっと,
  各班ごとにずいぶん\mura{違うことがチャレンジになる}
\item<2-> 大事なのは「知らなくてもナントカなる」という\mura{気合}
\item<3-> 全てを事前に知っておくなどということは\mura{不可能}
\item<4-> ましてや,\mura{「教えておく」のは無理}.
わからないことの答えが教科書に書いてあるというのは
無理だし無意味
\item<5-> 作業を効率的にするために,
  後々有効な方法やツールを腰を据えて学ぶ
  (その場しのぎで終わらせない), \mura{将来のためのレベルアップを目指す姿勢}
\item<6-> やる気のわく面白い変更・拡張の
  \mura{アイデアをチームで議論し,クラスでも議論}する.
  電車の中や寝ながらも\mura{何をしたいか考える}
\item<7-> だが時間的な制約もあるので
  \mura{ある程度で妥協\&決心}も必要(そこで萎えない)
\end{itemize}
\end{frame}

%%%%%%%%%%%%%%%%% 
\begin{frame}
  \frametitle{オススメの心構え(2)}
  \mura{オレ(私)はそんなに強くないし\ldots} という人へ
  \begin{itemize}
  \item<2-> いわゆる「強い」という印象 $\approx$ プログラムが書ける,
    バイトで稼いだり何やら本格的な開発が出来る, \ldots
  \item<3-> 正しいプログラムが書けるかどうか, そのためのきちんとした
    論理的・数学的説明が出来るか, という以外の部分は,
    それに比べれば\mura{「表面的・経験すればどうにかなる話」}
  \item<4-> バイト代は稼げなくても, オープソースソフトの開発に加わる,
    その手前の経験をこの演習でやる(その過程でツールや,
    スキルを身につける)ことで, 身につく話
  \item<5-> 個々の場面で行き当たりばったり, 答えを知れば良しとせず,
    \mura{今後使えるスキルや道具を手に入れよう}, という心構え
  \end{itemize}
\end{frame}

%%%%%%%%%%%%%%%%% 
\begin{frame}[fragile]
\frametitle{今日の残り時間の課題}
\begin{itemize}
\item gnuplotを以下のように変更する
\item キーワード``plot''を省略可能にする.例:
\begin{lstlisting}
gnuplot> sin(x)
\end{lstlisting}
だけで
\begin{lstlisting}
gnuplot> plot sin(x)
\end{lstlisting}
の動作をするようにする
\end{itemize}
\end{frame}

%%%%%%%%%%%%%%%%% 
\begin{frame}[fragile]
\frametitle{代案}
\begin{itemize}
\item gnuplotを以下のように変更する
\item 巨大データを間引く機能
\begin{lstlisting}
gnuplot> plot "hoge.dat" max_samples 10000
\end{lstlisting}
と書くと,"hoge.dat" に1億個の点が含まれていても,
10000だけを等確率で選び出し,表示する(メモリと時間の節約).
\end{itemize}
\end{frame}

%%%%%%%%%%%%%%%%% 
\begin{frame}
\frametitle{そのために練習して欲しい過程}
\begin{itemize}
\item gnuplotのソースファイルをダウンロード
\item gnuplotを,デバッグシンボル有り({\tt -g}),
  最適化ゼロ({\tt -O0})でビルド(configure, make)
\item gdbで動作追跡
\item 気が向いたら\texttt{htags --suggest}でソース一覧
\item ここまでは教科書に懇切丁寧なガイドあり
\item (Linux) uftraceを使ってトレースもしてみて(ホームページ$\rightarrow$
  拙著: ブログ記事 参照)
\item その他教科書の付録には
  \begin{itemize}
  \item 分割コンパイルのための規則
  \item makeってなにさ
  \item configureについて少し
  \end{itemize}
\item あとは臨機応変で!
\end{itemize}
\end{frame}


%%%%%%%%%%%%%%%%% 
\begin{frame}[fragile]
  \begin{itemize}
  \item その他独自案を考えてくれてもいいですが,
    今日の目的はあくまでソースからビルド,gdbで追いかける練習,
    ということなので,ハマる可能性のあることに挑戦しなくてもいいです
  \item ソフト・環境ごとに異なる地雷があるので,まずは穏便にgnuplot
    (こちらでお手軽なことを確認済み)をオススメ
  \item メインディッシュは次回以降,自分で決める
  \end{itemize}
\end{frame}

\end{document}

%%%%%%%%%%%%%%%%% 
\begin{frame}
\frametitle{我々のアクションアイテム?}

\begin{enumerate}
\item {\tiny 路頭に迷わないための} 候補ソフト案
\item {\tiny 路頭に迷わないための} 候補変更案
\item 実際にビルドしてみる
\item 道中,遭遇した問題の共有
\item 一般的に,何を予備知識・調査技法として持っておけば突破できるか?
  \begin{enumerate}
  \item 例: 共有ライブラリのパス({\tt LD\_LIBRARYPATH}や{\tt -Wl,-R})
  \item 例: gcc -E でマクロがどう展開されているか, 
    includeはどのファイルを読んでいるか
  \item コンパイル時間短縮のために,Makefileを読み解く方法?
  \item configure で失敗した時のシェルスクリプトの読み方?
  \end{enumerate}
\item そのような知見を我々が収集し,一般的に役に立つものは教えておくとか
  (教科書の拡充ないし,htmlでの補足説明)
\end{enumerate}
\end{frame}


%%%%%%%%%%%%%%%%% 
\begin{frame}
\frametitle{{\tiny 路頭に迷わないための}候補ソフト具体案}
\begin{itemize}
\item gcc
\item LLVM
\item Python
\item JavaScript
\item mruby
\item sqlite
\item Maria DB
\item postgresql
\item Linux
\item FreeBSD
\item gnumeric
\item inkscape
\item Chrome
\item Firefox
\item vlc
\item sublime
\item nautilus
\item evince
\item mikutter
\item lightdm
\end{itemize}
\end{frame}



%%%%%%%%%%%%%%%%% 
\begin{frame}
\frametitle{{\tiny 路頭に迷わないための}レベル0課題案}
\begin{itemize}
\item gcc
\item LLVM
\item Python
\item JavaScript
  \begin{itemize}
  \item 適当に console オブジェクトとかのメソッドを追加
  \end{itemize}
\item mruby
\item sqlite
  \begin{itemize}
  \item ``start transaction'' で beginの代わりになる
  \item (level 1) dump で他の形式に
  \end{itemize}
\item Maria DB
  \begin{itemize}
  \item ``.quit'' で終わる
  \end{itemize}
\item postgresql
  \begin{itemize}
  \item ``.quit'' で終わる
  \end{itemize}
\item Linux
  \begin{itemize}
  \item システムコール一個追加
  \item fd からファイル名を返す; パイプのバッファの大きさ
  \end{itemize}
% \item FreeBSD
\item Mozc
  \begin{itemize}
  \item 変な候補を常に表示
  \end{itemize}
\item gnumeric
  \begin{itemize}
  \item 組込み関数一個追加({\tt =sin}的な)
  \end{itemize}
\item inkscape
  \begin{itemize}
  \item アイコンの形を変える,図形の種類を増やす
  \end{itemize}
\item Chrome
  \begin{itemize}
  \item ポップアップの表示の仕方を変える
  \end{itemize}
\item Firefox
  \begin{itemize}
  \item web開発ツールの表示の仕方を変える(KBじゃなくて,とか); 遅いのを赤くする
  \end{itemize}
\item vlc
  \begin{itemize}
  \item raw形式で音声を取り出せ
  \end{itemize}
\item nautilus
\item evince
\item mikutter
\item lightdm
\end{itemize}
\end{frame}





\end{document}



