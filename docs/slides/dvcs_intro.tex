\documentclass[12pt,dvipdfmx]{beamer}
\usepackage{graphicx}
\usepackage{hhline}
\DeclareGraphicsExtensions{.pdf}
\DeclareGraphicsExtensions{.eps}
\graphicspath{{out/}{out/tex/}{out/tex/gpl/}{out/tex/svg/}{out/tex/dot/}}
% \graphicspath{{out/}{out/tex/}{out/pdf/}{out/eps/}{out/tex/gpl/}{out/tex/svg/}{out/pdf/dot/}{out/pdf/gpl/}{out/pdf/img/}{out/pdf/odg/}{out/pdf/svg/}{out/eps/dot/}{out/eps/gpl/}{out/eps/img/}{out/eps/odg/}{out/eps/svg/}}
\usepackage{listings}
\usepackage{fancybox}
\usepackage{hyperref}
\usepackage{color}

%%%%%%%%%%%%%%%%%%%%%%%%%%%
%%% themes
%%%%%%%%%%%%%%%%%%%%%%%%%%%
\usetheme{Szeged}
\setbeamertemplate{headline}{}
%% no navigation bar
% default boxes Bergen Boadilla Madrid Pittsburgh Rochester
%% tree-like navigation bar
% Antibes JuanLesPins Montpellier
%% toc sidebar
% Berkeley PaloAlto Goettingen Marburg Hannover Berlin Ilmenau Dresden Darmstadt Frankfurt Singapore Szeged
%% Section and Subsection Tables
% Copenhagen Luebeck Malmoe Warsaw

%%%%%%%%%%%%%%%%%%%%%%%%%%%
%%% innerthemes
%%%%%%%%%%%%%%%%%%%%%%%%%%%
\useinnertheme{circles}	% default circles rectangles rounded inmargin

%%%%%%%%%%%%%%%%%%%%%%%%%%%
%%% outerthemes
%%%%%%%%%%%%%%%%%%%%%%%%%%%
% outertheme
% \useoutertheme{default}	% default infolines miniframes smoothbars sidebar split shadow tree smoothtree

%%%%%%%%%%%%%%%%%%%%%%%%%%%
%%% colorthemes
%%%%%%%%%%%%%%%%%%%%%%%%%%%
%\usecolortheme{seahorse}
\usecolortheme{seagull}
%% special purpose
% default structure sidebartab
%% complete
% albatross beetle crane dove fly seagull
%% inner
% lily orchid rose
%% outer
% whale seahorse dolphin

\definecolor{bg-dark}{rgb}{0.6784313725490196, 0.8156862745098039, 0.5333333333333333}
\definecolor{bg-darker}{rgb}{0.5686274509803921, 0.6745098039215687, 0.3686274509803922}
\definecolor{bg-light}{rgb}{0.9490196078431372, 0.984313725490196, 0.9372549019607843}

\mode<presentation>
 {
 \setbeamercolor*{palette secondary}{use=structure,fg=black,bg=bg-dark}
 \setbeamercolor*{palette tertiary}{use=structure,fg=black,bg=bg-darker}
 }


%%%%%%%%%%%%%%%%%%%%%%%%%%%
%%% fontthemes
%%%%%%%%%%%%%%%%%%%%%%%%%%%
\usefonttheme{structurebold}
% default professionalfonts serif structurebold structureitalicserif structuresmallcapsserif

%%%%%%%%%%%%%%%%%%%%%%%%%%%
%%% generally useful beamer settings
%%%%%%%%%%%%%%%%%%%%%%%%%%%
%
\AtBeginDvi{\special{pdf:tounicode EUC-UCS2}}
% do not show navigation
\setbeamertemplate{navigation symbols}{}
% show page numbers
\setbeamertemplate{footline}[frame number]
\setbeamersize{description width=0.57cm}

%%%%%%%%%%%%%%%%%%%%%%%%%%%
%%% define some colors for convenience
%%%%%%%%%%%%%%%%%%%%%%%%%%%

\newcommand{\mido}[1]{{\color{green}#1}}
\newcommand{\mura}[1]{{\color{purple}#1}}
\newcommand{\ore}[1]{{\color{orange}#1}}
\newcommand{\ao}[1]{{\color{blue}#1}}
\newcommand{\aka}[1]{{\color{red}#1}}

\setbeamercolor{ex}{bg=cyan!20!white}

%%%%%%%%%%%%%%%%%%%%%%%%%%%
%%% how to typset code
%%%%%%%%%%%%%%%%%%%%%%%%%%%

\lstset{
language = C,
 backgroundcolor=\color{bg-light},
% numbers = left,
% numberstyle = {\tiny \emph},
% numbersep = 10pt,
breaklines = true,
breakindent = 40pt,
frame = single,
%frameround = ffft,
%framesep = 3pt,
rulesep = 1pt,
rulecolor = {\color{bg-darker}},
rulesepcolor = {\color{bg-darker}},
flexiblecolumns = true,
keepspaces = true,
basicstyle = \ttfamily\scriptsize,
identifierstyle = ,
commentstyle = ,
stringstyle = ,
showstringspaces = false,
tabsize = 4,
escapechar=\@,
}


\AtBeginSection[]{
  \begin{frame}
  \vfill
  \centering
  \begin{beamercolorbox}[sep=8pt,center,shadow=true,rounded=true]{title}
    \usebeamerfont{title}\secname\par%
  \end{beamercolorbox}
  \vfill
  \end{frame}
}

\usebackgroundtemplate{%
  \includegraphics[width=\paperwidth,height=\paperheight]{images/bg.png}}

\title{複数人での開発の方法}
\author{TA: 藤澤 (B4)}
\date{2017/10/24}

\begin{document}
\maketitle

\section{目的}

%%%%%%%%%%%%%%%%%
\begin{frame}[containsverbatim]
\frametitle{目的}

とりあえずこの講義で目的とする範囲

\begin{itemize}
\item 1つのソースコードを対象に{\Large 複数人}で{\Large 同時}に開発したい
\item 全員の成果を結合し、{\Large 共有}したい
\item そのための余計な手間は避けたい
\end{itemize}
\end{frame}

\section{実現方法}

%%%%%%%%%%%%%%%%%
\begin{frame}[containsverbatim]
\frametitle{目的の分解 (1/3)}

目的をより細かく見ると

\begin{itemize}
\item 手元であるファイルを{\Large 変更}する
\item その{\Large 変更}を他の人と{\Large 共有}する \\
 = 相手の{\Large 変更}が手元に{\Large 反映}される
\end{itemize}

\begin{center}
  \includegraphics[width=1\hsize]{images/purpose-1.png}
\end{center}
\end{frame}

%%%%%%%%%%%%%%%%%
\begin{frame}[containsverbatim]
\frametitle{目的の分解 (2/3)}

{\Large 変更}とは何か

\begin{center}
「元の状態」から「何」を「どう」変えたか
\end{center}

\begin{center}
  \includegraphics[width=1\hsize]{images/purpose-2.png}
\end{center}
\end{frame}

%%%%%%%%%%%%%%%%%
\begin{frame}[containsverbatim]
\frametitle{目的の分解 (3/3)}

{\Large 共有}とは何か

\begin{center}
「変更」を相手に伝えて、相手側で「再現」させる
\end{center}

\begin{center}
  \includegraphics[width=1\hsize]{images/purpose-3.png}
\end{center}
\end{frame}

%%%%%%%%%%%%%%%%%
\begin{frame}[containsverbatim]
\frametitle{例 - お互いに変更しあう (1/3)}

\begin{center}
それぞれバラバラに変更しても
\end{center}

\begin{center}
  \includegraphics[width=1\hsize]{images/2-clients-1.png}
\end{center}
\end{frame}

%%%%%%%%%%%%%%%%%
\begin{frame}[containsverbatim]
\frametitle{例 - お互いに変更しあう (2/3)}

\begin{center}
変更を伝えれば
\end{center}

\begin{center}
  \includegraphics[width=1\hsize]{images/2-clients-2.png}
\end{center}
\end{frame}

%%%%%%%%%%%%%%%%%
\begin{frame}[containsverbatim]
\frametitle{例 - お互いに変更しあう (3/3)}

\begin{center}
両方の変更が正しく反映される
\end{center}

\begin{center}
  \includegraphics[width=1\hsize]{images/2-clients-3.png}
\end{center}
\end{frame}

%%%%%%%%%%%%%%%%%
\begin{frame}[containsverbatim]
\frametitle{大勢で変更しあう}

\begin{center}
大勢になったら、全員がお互いにやりとりしてられない
\end{center}

\begin{center}
  \includegraphics[width=1\hsize]{images/many-clients-1.png}
\end{center}
\end{frame}

%%%%%%%%%%%%%%%%%
\begin{frame}[containsverbatim]
\frametitle{中間の置き場の導入}

\begin{center}
中間の置き場に変更を貯めて、各自そこから取ってくる
\end{center}

\begin{center}
  \includegraphics[width=1\hsize]{images/many-clients-2.png}
\end{center}
\end{frame}

\section{ツールによる実現}

%%%%%%%%%%%%%%%%%
\begin{frame}[containsverbatim]
\frametitle{使えそうなツール}

世の中にはこういう事を実現するための\\
いろんなツールがありますが、今回は

\begin{itemize}
\item Git
\item GitLab
\end{itemize}

これを見てみます

\end{frame}

%%%%%%%%%%%%%%%%%
\begin{frame}[containsverbatim]
\frametitle{使えそうなツール - Git}

{\Large Git とは}

\begin{itemize}
\item 分散バージョン管理システム
\item 機能/特徴
  \begin{itemize}
  \item バージョン管理 $\Leftarrow$ 「変更」の管理
  \item スナップショットベース {\scriptsize (変更ベースとはやや違うけどオッケー)}
  \item 分散型 $\Leftarrow$ 「共有」
  \item その他いろいろ
  \end{itemize}
\end{itemize}
\end{frame}

%%%%%%%%%%%%%%%%%
\begin{frame}[containsverbatim]
\frametitle{使えそうなツール - GitLab}

{\Large GitLab とは}

\begin{itemize}
\item ソフトウェア開発支援環境
\item 機能/特徴
  \begin{itemize}
  \item Git ベースのソースコード管理
  \item プロジェクト(=リポジトリ)管理 $\Leftarrow$ 「中間の置き場」
  \item ユーザ/グループ管理
  \item Issue 管理
  \item マージリクエストによるコードレビュー
  \item その他いろいろ
  \end{itemize}
\end{itemize}
\end{frame}

\section{実践}

%%%%%%%%%%%%%%%%%
\begin{frame}[containsverbatim]
\frametitle{実践}

\begin{center}
実際にやってみよう

\vspace{5mm}

{\small \url{https://doss.eidos.ic.i.u-tokyo.ac.jp/html/git.html}}
\end{center}
\end{frame}

\end{document}
